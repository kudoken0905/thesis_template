\chapter{序論}
\section{背景}
近年高齢者を始めとする自動車による交通事故が問題視されている。
そして高齢者の中で多い要因がブレーキとアクセルの踏み間違いなどの操作不適と安全不確認の二つである。
このような問題を解決するために2021年11月以降の新型乗用車に衝突被害軽減ブレーキの装備が義務付けられることとなり、
また65歳以上の高齢者を対象に安全運転サポート車普及促進事業補助金、通称「サポカー補助金」等の実施がされてきている。
しかし、サポカー補助金は2021年11月29日に申請受付の終了されており、自動ブレーキ機能も後付けすることはできない。

このように高齢者による交通事故を減らすために様々な施策はされてきてはいるが、いずれも新型車を購入する必要があり、非常にコストが高くなってしまっている。
このような問題を解決するために操作不適による交通事故の抑制にはアクセルとブレーキの踏み間違いを監視し警告と加速の抑制を行う装置が、
安全不確認による交通事故の抑制には前方監視し前方の車への急接近や歩行者を検知すると警告する装置がそれぞれ後付けで設置できるように製品化されている。
どちらも新型車への乗り換えよりはコストが抑えられているもののそれでもまだ多くの人が設置するには至っていないことから安全装置の設置を促すためには更にコストを抑える必要がある。

特に安全不確認の抑制に対しての警告装置は画像認識を用いて行うことができるため、
スマートフォンなどのカメラ付き端末に実装することができれば価格的コストも装置の設置にかかる手間も軽減されるため課題を解決できるのではないか。

\section{先行事例}

安全不確認の抑制には道路標識や人の認識が必要となる。
そのため、










% dvipdfmxとhereのテスト
%\begin{figure}[H]
%	\begin{center}
%		\includegraphics[width=1.0\linewidth]{../zero.png}
%		\caption{}
%		\label{fig:}
%	\end{center}
%\end{figure}
%
