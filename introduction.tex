\chapter{序論}
\section{背景}
近年, 警備ロボットや清掃ロボットとして, 自律移動ロボットが利用される機会が増えている. 
%%%%採用->利用かな?
自律移動ロボットとはセンサやカメラで周辺環境を認識し, 自己位置推定をして自律走行をするロボットである. 
%@@@自己位置推定しない奴もいます。
%@@@「自己位置推定」の説明をしていないんですが、サボってもいいのでその場合は引用を。早めにやっておきましょう。

基本的な自己位置推定の手法として, モンテカルロ位置推定(以下MCL)があげられる. 
MCLは, 座標と向きの情報を持つパーティクルの分布で, ロボットの自己位置の確率分布を近似し, 自己位置を推定するアルゴリズムである.
MCLで利用されるパーティクルフィルタは, ロボットの移動誤差を考慮してパーティクルを動かすため, 小石や地面の凹凸といったノイズに強いという特徴がある. 
%@@@んー。ちょっとちがう。そもそも「強い」って何と比べて?比較対象がないのにそう言えるでしょうか?
しかし, 数百から数千のパーティクルを用いて計算をするため, 計算量が多いという欠点もある.
%%%%ここらへんでMCLの利点や欠点について説明しましょう。
%@@@他の自己位置推定手法の話も入ってきます。
%@@@これも何と比べて?

MCLは, ロボットの移動誤差を考慮して推定するため, パーティクルが広がり, 複数のクラスタへ分離することがある. 
パーティクルが分離してしまうとロボットのいないパーティクルのクラスタができることになる. 
その後, ロボットのいる方のクラスタが消えてしまうと, 自己位置を見失うことになり誘拐状態になってしまう. 
誘拐状態とは, ロボットの真の位置とは異なる位置を, 自己位置として推定してしまう状態のことである.
真の位置にパーティクルが無くなる誘拐状態では, パーティクルの分布で自己位置の確率分布を近似するMCLで解消することは困難である. 
%@@@この段落に書いてあることは説明不足で、たとえば米田研の人に説明すると仮定すると、わかんないと思います。図とかで説明を。


\section{従来研究}
%@@@たぶん分かってると思いますが、まだ従来研究の節になってないですね。

そこで, 誘拐状態になる前にパーティクルが複数クラスタへ分離したことを検知し, 分離を解消する行動を取ることで, 誘拐状態にならずに自己位置推定が成功すると考えられる. 
%@@@「そこで」というのは安易な言葉選びだと思います。「そこで」は前の文が理由になってないといけないのですが、理由になってません。
検知する方法として自律走行中にクラスタ数の自動推定アルゴリズムで推定し続ける方法が考えられる. 
%@@@説明が雑。

本論文では, 誘拐状態になる前に, MCLにおけるパーティクルの複数のクラスタへの分離を検知する方法を提案する. 
%@@@段落構成が雑。1文1段落










------------------------


上田\index{うえだ@上田}は、いろいろ書いているが、あまり引用されない。
例えば、\cite{上田2015gihyo,ueda2015,上田2015jsai}
がある。

\ref{chap:purpose}章で目的を述べる。

% dvipdfmxとhereのテスト
%\begin{figure}[H]
%	\begin{center}
%		\includegraphics[width=1.0\linewidth]{../zero.png}
%		\caption{}
%		\label{fig:}
%	\end{center}
%\end{figure}
%
