\chapter{研究の目的}\label{chap:purpose}

そこで, 本論文では, MCLのパーティクルのクラスタ数を推定し, パーティクルの分離を検知する方法を提案する. 
クラスタ数の推定は, クラスタ数の自動推定アルゴリズムを用いる.
そして, 推定クラスタ数が2以上になったとき, パーティクルが分離したと判断する.

また, 分離を検知した後, 分離を解消する行動を取らせるため, リアルタイムで分離を検知しなければならない.
ここでは, 計算時間がロボットに搭載されたセンサのスキャン周期より短いことが要求される. 
なぜなら, センサのデータが更新されるたび, パーティクルをリセットするかどうかの判定されるからである.
リセットとは, 誘拐状態のように, センサからの情報とロボットの推定結果がずれているときに, 自動でパーティクルを置き直すことである.

実験ではHOKUYOのUTM-30LXを使用したため, 計算時間を25ms以内にすることを目標とした.

---
そこで、上田の研究をもっと時代におもねった方法に変える
手法の研究を行う。

