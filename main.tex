\documentclass[a4paper,11pt]{jsbook}

\newcommand{\V}[1]{\boldsymbol{#1}}
\def\thline{\noalign{\hrule height 1pt}}
\def\tvline{\vrule width 1pt}

\usepackage{here} %図の場所の指定で[H](ここに貼る)を指定するためのパッケージ
\usepackage{makeidx}
\usepackage{amsmath}
\usepackage{amssymb}
\usepackage[dvipdfmx]{graphicx} %dvipdfmxはjpgやpngの張り込みのために使用

\makeindex

\pagenumbering{roman}

\begin{document}
% 表紙
\title{平成27年度 卒業論文\\
ドリルキングアンセムの研究}

\author{鳥"留噛男 \\
China Institute of Technology}

\date{2017年2月x日}

\maketitle

%%% 但し書き等 %%%
この論文は、読んだあと自動的に消滅する。
\clearpage

%%% 献辞 %%%
% D論、あるいは誰かを亡くしたときの卒論、修論等で
% 配偶者や配偶者の予定となる人の名前は覚悟をもって書くこと
\thispagestyle{empty}
\vfil
\ \\
\vspace{15em}
\begin{center}
	{\Large 最愛の京成線に捧ぐ }
\end{center}
% 献辞をかかない場合はここまでコメントアウト

\include{preface}

\tableofcontents

%\cleardoublepage

%%% 本文 %%%
% 章のページの先頭は左側(奇数ページ)に来る

\cleardoublepage
\pagenumbering{arabic}

\chapter{序論}
\section{背景}
近年高齢者を始めとする自動車による交通事故が問題視されている。
そして高齢者の中で多い要因がブレーキとアクセルの踏み間違いなどの操作不適と安全不確認の二つである。
このような問題を解決するために2021年11月以降の新型乗用車に衝突被害軽減ブレーキの装備が義務付けられることとなり、
また65歳以上の高齢者を対象に安全運転サポート車普及促進事業補助金、通称「サポカー補助金」等の実施がされてきている。
しかし、サポカー補助金は2021年11月29日に申請受付の終了されており、自動ブレーキ機能も後付けすることはできない。

このように高齢者による交通事故を減らすために様々な施策はされてきてはいるが、いずれも新型車を購入する必要があり、非常にコストが高くなってしまっている。
このような問題を解決するために操作不適による交通事故の抑制にはアクセルとブレーキの踏み間違いを監視し警告と加速の抑制を行う装置が、
安全不確認による交通事故の抑制には前方監視し前方の車への急接近や歩行者を検知すると警告する装置がそれぞれ後付けで設置できるように製品化されている。
どちらも新型車への乗り換えよりはコストが抑えられているもののそれでもまだ多くの人が設置するには至っていないことから安全装置の設置を促すためには更にコストを抑える必要がある。

特に安全不確認の抑制に対しての警告装置は画像認識を用いて行うことができるため、
スマートフォンなどのカメラ付き端末に実装することができれば価格的コストも装置の設置にかかる手間も軽減されるため課題を解決できるのではないか。

\section{先行事例}

安全不確認の抑制には道路標識や人の認識が必要となる。
そのため、










% dvipdfmxとhereのテスト
%\begin{figure}[H]
%	\begin{center}
%		\includegraphics[width=1.0\linewidth]{../zero.png}
%		\caption{}
%		\label{fig:}
%	\end{center}
%\end{figure}
%

\chapter{研究の目的}\label{chap:purpose}

\chapter{提案手法}\label{chap:method}:


\section{手法の概要}

\section{手法の実装}









----------

図に書くと図\ref{fig:vq_map_128part}っていう感じ。
式で書くとだいたい以下のような感じになるんじゃないんかなー。
式(\ref{eq:j})が肝。

\begin{figure}[h]
        \begin{center}
        \includegraphics[width=1.0\linewidth]{figs/vq_map_128part.eps}
        \caption{Representative Vectors of the $N_c = 128$ Map}
        \label{fig:vq_map_128part}
        \end{center}
\end{figure}




\begin{align}
s_0, a(t_0), s(t_1), a(t_1), s(t_2), a(t_2), \dots, a(t_{T-1}), s_\text{f} \quad (s_0 = s(t_0), s_\text{f} = s(t_T)). 
\end{align}
\begin{align}
& s_0, \pi(s_0), s(t_1), \pi(s(t_1)), s(t_2), \pi(s(t_2)), \dots, \pi(s(t_{T-1})), s_\text{f}
\end{align}
\begin{align}
\pi &: \mathcal{S} \to \mathcal{A} \label{eq:policy_state_action_sequence}
\end{align}
\begin{align}
\mathcal{S} &= \{s_i | i=0,1,2,\dots,N-1 \}, \text{ and} \\
\mathcal{A} &= \{a_j | j=0,1,2,\dots,M-1 \}
\end{align}
\begin{align}
\pi : \mathcal{S} - \mathcal{S}_\text{f} \to \mathcal{A}. \label{eq:policy}
\end{align}
\begin{align}
\dot{\V{x}}(t) &= \V{f}[\V{x}(t),\V{u}(t)], \quad \V{x}(0) = \V{x}_0, \quad t \in [0,t_\text{f}].\label{eq:system} \\
&  \nonumber 
\end{align}
\begin{align}
g[\V{x}(t), \V{u}(t)] \in \Re \quad (t \in [0,t_\text{f}]). \label{eq:evaluation_function}
\end{align}
\begin{align}
J[\V{u}] = \int_{0}^{t_\text{f}} g[\V{x}(t), \V{u}(t)] dt + V(\V{x}_\text{f}).  \label{eq:functional}
\end{align}
\begin{align}
\max_{\V{u}:[0,t_\text{f}) \to \Re^m} J[\V{u};\V{x}_0].  \label{eq:optimal_control_problem}
\end{align}
\begin{align}
\V\pi^*: \Re^n \to \Re^m
\end{align}
\begin{align}
\max_{\V{u}:[0,t_\text{f}) \to \Re^m} J[\V{u};\V{x}_0] &= \max_{\V{u}:[0,t') \to \Re^m} \int_{0}^{t'} g[\V{x}(t), \V{u}(t)] dt \nonumber \\ &+ \max_{\V{u}:[t',t_\text{f}) \to \Re^m} \int_{t'}^{t_\text{f}} g[\V{x}(t), \V{u}(t)] dt + V(\V{x}_\text{f}) \nonumber \\
	&= \max_{\V{u}:[0,t') \to \Re^m} \int_{0}^{t'} g[\V{x}(t), \V{u}(t)] dt + \max_{\V{u}:[t',t_\text{f}) \to \Re^m} J[\V{u};\V{x}(t')]. \label{eq:j}
\end{align}
\begin{align}
V^{\V\pi}(\V{x}) &= J[\V{u};\V{x}], \label{eq:def_of_value} \\
&\text{ where } \V{u}(t) = \V\pi(\V{x}(t)), \ 0\le t \le t_\text{f}. \nonumber 
\end{align}
\begin{align}
\mathcal{P}_{ss'}^a &= P[s(t_{i+1}) = s' | s(t) = s,a(t) = a], \label{eq:state_transition}\\
&(\forall t \in \{t_0,t_1,\dots,t_{T-1}\}, \forall s \in \mathcal{S} - \mathcal{S}_\text{f}, \text{ and } \forall s' \in \mathcal{S}). \nonumber
\end{align}
\begin{align}
\mathcal{R}_{ss'}^a \in \Re
\end{align}
\begin{align}
J[a;s(t_0)] = J[a(0),a(1),\dots,a(t_{T-1})] = \sum_{i=0}^{T-1} \mathcal{R}_{s(t_i)s(t_{i+1})}^{a(t_i)} + V(s(t_T)),
\end{align}
\begin{align}
\max J[a;s(t_0)]. 
\end{align}
\begin{align}
J^{\V{\pi}} = \int_\mathcal{X} p(\V{x}_0) J[\V{u};\V{x}_0] d\V{x}_0 \quad\Big(\V{u}(t) = \V\pi(\V{x}(t))\Big),\label{eq:eval_general}
\end{align}
\begin{align}
\dfrac{\partial V(\V{x})}{\partial t} = \max_{\V{u}\in\mathcal{U}} \left[g[\V{x},\V{u}] + \dfrac{\partial V(\V{x})}{\partial\V{x}} \V{f}[\V{x},\V{u}] \right].\label{eq:hjb}
\end{align}
\begin{align}
U_\text{att}(\V{x}) = \dfrac{1}{2} \xi \rho^2 (\V{x})
\end{align}
\begin{align}
U_\text{rep}(\V{x}) =
\begin{cases}
\dfrac{1}{2}\eta \left( \dfrac{1}{\rho(\V{x})} - \dfrac{1}{\rho_0} \right)^2 &\text{if } \rho(\V{x}) \le \rho_0, \\
0 &\text{if } \rho(\V{x}) > \rho_0,
\end{cases}
\end{align}
\begin{align}
U(\V{x}) = U_\text{att}(\V{x}) + U_\text{rep}(\V{x})
\end{align}
\begin{align}
\V{F}(\V{x}) &= - (\partial U/\partial x_1,\partial U/\partial x_2,\dots,\partial U/\partial x_n)^T \nonumber \\
&= -\nabla U({\V{x}}). 
\end{align}
\begin{align}
V(\V{x};\theta_1,\theta_2,\dots,\theta_{N_\theta}) \nonumber
\end{align}
\begin{align}
\phi_i(\V{x}) &= \exp \left\{-\dfrac{1}{2}(\V{x} - \V{c}_i)^t M_i (\V{x} - \V{c}_i) \right\},
\end{align}
\begin{align}
b_i(\V{x}) &= \dfrac{\phi_i(\V{x})}{\sum_{j=1}^{N_\phi} \phi_j(\V{x})}, \ (N_\phi: \text{ number of RBFs in the space})
\end{align}
\begin{align}
V(\V{x}) &= \sum_{i=1}^{N_\phi} \nu_i b_i(\V{x}). \label{eq:rbf_weighted_sum}
\end{align}
\begin{align}
\phi_i(x) &= \exp \left\{-\dfrac{1}{2}(x - i)^2 \right\} \nonumber
\end{align}

\begin{align}
V(\V{x}) = \sum_{i=0}^3 w_i V(\V{x}_i)
\end{align}


\begin{table}[htbp]
        \begin{center}
	\caption{謎のパラメータ}
        \label{table:parameter_value}
\begin{footnotesize}
\begin{minipage}{12em}
        \begin{tabular}{c|rl}
\multicolumn{3}{c}{(a)}\\
        \thline
parameter & \multicolumn{2}{c}{value} \\
        \hline
$\ell_1,\ell_2$ & 1.0&[m] \\
$\ell_{c1},\ell_{c1}$ & 0.50&[m] \\
$m_1,m_2$ & 1.0&[kg] \\
$I_1,I_2$ & 1.0&[kg m$^2$] \\
$g$ & 9.8&[m/s$^2$] \\
   \thline
  \end{tabular}
\end{minipage}
\hspace{2em}
\begin{minipage}{12em}
        \begin{tabular}{c|l}
\multicolumn{2}{c}{(b)}\\
        \thline
variable & domain \\
        \hline
$\theta_1$ & $(-\infty,\infty)$ \\
$\theta_2$ & $(-\infty,\infty)$ \\
$\dot\theta_1$ & $[-720,720]$[deg/s] \\
$\dot\theta_2$ & $[-1620,1620]$[deg/s] \\
\hline
$\tau$ & $-1,0,$ or $1$[Nm] \\
   \thline
  \end{tabular}
\end{minipage}
\end{footnotesize}
  \end{center}
\end{table}



\include{conclusion}

\appendix
\include{appendix}

%%% 参考文献 %%%
% よほどのことが無い限りet al.は使わないことにしましょう
\bibliographystyle{jualpha}
\bibliography{./references}

\newpage
\printindex

\end{document}


