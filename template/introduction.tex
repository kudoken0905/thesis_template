\chapter{序論}
\section{背景}
近年, 警備ロボットや清掃ロボットとして, 自律移動ロボットが利用される機会が増えている. 
%%%%採用->利用かな?
自律移動ロボットとはセンサやカメラで周辺環境を認識し, 自己位置推定をして自律走行をするロボットである. 

基本的な自己位置推定の手法として, モンテカルロ位置推定(以下MCL)があげられる. 
MCLは, 座標と向きの情報を持つパーティクルの分布で, ロボットの自己位置の確率分布を近似し, 自己位置を推定するアルゴリズムである.
%@@@ここらへんでMCLの利点や欠点について説明しましょう。他の自己位置推定手法の話も入ってきます。

MCLは, ロボットの移動誤差を考慮して推定するため, パーティクルが広がり, 複数のクラスタへ分離することがある. 
パーティクルが分離してしまうとロボットのいないパーティクルのクラスタができることになる. 
%%%%ここで段落はわけないほうがいいと思います。
その後, ロボットのいる方のクラスタが消えてしまうと, 自己位置を見失うことになり誘拐状態になってしまう. 
誘拐状態とは, ロボットの真の位置とは異なる位置を, 自己位置として推定してしまう状態のことである.
真の位置にパーティクルが無くなる誘拐状態では, パーティクルの分布で自己位置の確率分布を近似するMCLで解消することは困難である. 

%@@@ここで、「1.2 従来研究」に移行して、他の研究事例を説明しましょう。

そこで, 誘拐状態になる前にパーティクルが複数クラスタへ分離したことを検知し, 分離を解消する行動を取ることで, 誘拐状態にならずに自己位置推定が成功すると考えられる. 
検知する方法として自律走行中にクラスタ数の自動推定アルゴリズムで推定し続ける方法が考えられる. 

本論文では, 誘拐状態になる前に, MCLにおけるパーティクルの複数のクラスタへの分離を検知する方法を提案する. 









------------------------


上田\index{うえだ@上田}は、いろいろ書いているが、あまり引用されない。
例えば、\cite{上田2015gihyo,ueda2015,上田2015jsai}
がある。

\ref{chap:purpose}章で目的を述べる。

% dvipdfmxとhereのテスト
%\begin{figure}[H]
%	\begin{center}
%		\includegraphics[width=1.0\linewidth]{../zero.png}
%		\caption{}
%		\label{fig:}
%	\end{center}
%\end{figure}
%
